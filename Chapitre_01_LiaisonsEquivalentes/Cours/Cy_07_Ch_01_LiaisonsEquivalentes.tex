\documentclass[10pt,fleqn]{article} % Default font size and left-justified equations
\usepackage[%
    pdftitle={Equilibrage des solides en rotation},
    pdfauthor={Xavier Pessoles}]{hyperref}

\input{style/new_style}
\input{style/macros_SII}
\usepackage{bm}
\fichetrue
\fichefalse

\proftrue
%\proffalse

%\tdtrue
\tdfalse

\courstrue
%\coursfalse



% -------------------------------------
% Déclaration des titres
% -------------------------------------

\def\discipline{Sciences \\Industrielles de \\ l'Ingénieur}
\def\xxtete{Sciences Industrielles de l'Ingénieur}

\def\classe{\textsf{PSI$\star$ -- MP}}
\def\xxnumpartie{Cycle 07}
\def\xxpartie{Modélisation des chaînes de solides dans le but de déterminer les contraintes géométriques dans les mécanismes}

\def\xxnumchapitre{Chapitre 1 \vspace{.2cm}}
\def\xxchapitre{\hspace{.12cm} Détermination des liaisons équivalentes}

\def\xxposongletx{2}
\def\xxposonglettext{1.45}
\def\xxposonglety{19}%16

\def\xxonglet{\textsf{Cycle 07}}

\def\xxactivite{Cours}
\def\xxauteur{\textsl{Xavier Pessoles}}

\def\xxcompetences{%
\textsl{%
\textbf{Savoirs et compétences :}\\
\begin{itemize}[label=\ding{112},font=\color{ocre}] 
\item \textit{Mod2.C34} : chaînes de solides;
\item \textit{Mod2.C34} : degré de mobilité du modèle;
\item \textit{Mod2.C34} : degré d’hyperstatisme du modèle;
\item \textit{Mod2.C34.SF1} : déterminer les conditions géométriques associées à l’hyperstatisme;
\item \textit{Mod2.C34} : résoudre le système associé à la fermeture cinématique et en déduire le degré de mobilité et d’hyperstatisme.
\end{itemize}
}}
		



\def\xxfigures{
\includegraphics[width=0.6\textwidth]{images/lola}\\
\textit{Robot humanoïde Lola}

\vspace{.5cm}

\includegraphics[width=0.5\textwidth]{images/simu}\\
\textit{Simulateur de vol Lockheed Martin}

}%figues de la page de garde

\def\xxpied{%
Cycle 07 -- Modélisation des chaînes de solides dans le but de déterminer les contraintes géométriques dans les mécanismes\\% afin de valider leurs performances.\\
Chapitre 1 -- \xxactivite%
}

\setcounter{secnumdepth}{5}
%---------------------------------------------------------------------------


\begin{document}
\chapterimage{png/Fond_CIN}
\input{style/new_pagegarde}
\setlength{\columnseprule}{.1pt}

\vspace{2cm}
\pagestyle{fancy}
\thispagestyle{plain}
%%%%%%%%%%%%%%%%%%%%%%%%%%%%%%%%%%%ù




\section{Introduction}
\subsection{Rappel sur les torseurs des liaisons}
\begin{defi}
De manière générale, le torseur cinématique peut être noté :
$$
\torseurcin{V}{i}{j}
=\torseurl{\vecto{i}{j}}{\vectv{P}{i}{j}}{P}
=\torseurl{p_{ij}\vect{x}+q_{ij}\vect{y}+r_{ij}\vect{z}}{u_{ij}\vect{x}+v_{ij}\vect{y}+w_{ij}\vect{z}}{P}
=\torseurcol{p_{ij}}{q_{ij}}{r_{ij}}{u_{ij}}{v_{ij}}{w_{ij}}{P,\mathcal{R}}.
$$

\textbf{On notera $n_c$ le nombre d'inconnues cinématiques d'une liaison.} En d'autres termes, $n_c$ correspond donc au nombre de mobilités de la liaison.
\end{defi}

\begin{defi}
De manière générale, le torseur statique peut être noté :
$$
\torseurstat{T}{i}{j}
=\torseurl{\vectf{i}{j}}{\vectm{P}{i}{j}}{P}
=\torseurl{X_{ij}\vect{x}+Y_{ij}\vect{y}+Z_{ij}\vect{z}}{L_{ij}\vect{x}+M_{ij}\vect{y}+N_{ij}\vect{z}}{P}
=\torseurcol{X_{ij}}{Y_{ij}}{Z_{ij}}{L_{ij}}{M_{ij}}{N_{ij}}{P,\mathcal{R}}.
$$

\textbf{On notera $n_s$ le nombre d'inconnues statiques d'une liaison.} En d'autres termes, $n_s$ correspond au degré de liaison. On a $n_s=6-n_c$.
\end{defi}
\subsection{Graphe des liaisons}

\begin{defi}
Selon la forme du graphe de liaisons, on peut distinguer 3 cas :
\begin{multicols}{3}
\begin{center}
\textbf{Les chaînes ouvertes} 
\end{center}

\begin{center}
\includegraphics[height=1.5cm]{images/ericc_01}

\vspace{.5cm}

\includegraphics[height=1.5cm]{images/ericc_02}
\end{center}

\begin{center}
\textbf{Les chaînes fermées} 
\end{center}

\begin{center}
\includegraphics[height=1.5cm]{images/sympact_01}

\vspace{.5cm}

\includegraphics[height=1.5cm]{images/sympact_02}
\end{center}

\begin{center}
\textbf{Les chaînes complexes} 
\end{center}

\begin{center}
\includegraphics[height=1.5cm]{images/haptique_01}

\vspace{.5cm}

\includegraphics[height=1.5cm]{images/haptique_02}
\end{center}

\end{multicols}

On appelle cycle, un chemin fermé ne passant pas deux fois par le même sommet.
À partir d’un graphe des liaisons donné, il est possible de vérifier qu’il existe un nombre
maximal de cycles indépendants. Ce nombre est appelé nombre cyclomatique. 

\textbf{En notant $L$ le nombre de liaisons et $S$ le nombre de solides, on note $\gamma$ le nombre cyclomatique et on a : $\gamma = L - S + 1$.}
\end{defi}

\begin{rem}
\begin{itemize}
\item Dans le cas d'une chaîne ouverte, $\gamma$ est nul. 
\item À partir du graphe de structure, il est possible de déterminer le nombre cyclomatique d'une chaîne complexe... si elle n'est pas trop complexe.
\end{itemize}
\end{rem}

\section{Liaisons équivalentes}
\begin{obj}
La détermination de la liaison équivalente correspondant à l'association de plusieurs liaisons doit permettre : 
\begin{itemize}
\item de transmettre les mêmes actions mécaniques que l'association de liaisons;
\item d'autoriser les mêmes mouvements relatifs que l'association de liaisons.
\end{itemize}
\end{obj}
\subsection{Liaisons en parallèles}
\subsection{Liaisons en série}

\end{document}




