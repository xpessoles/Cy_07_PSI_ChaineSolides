\documentclass[10pt,fleqn]{article} % Default font size and left-justified equations
\usepackage[%
    pdftitle={Modélisation dynamique},
    pdfauthor={Xavier Pessoles}]{hyperref}

    
\input{style/new_style}
\input{style/macros_SII}
\usepackage{multicol}
\usepackage{siunitx}
%\usepackage{picins}
\fichetrue
%\fichefalse

\proftrue
\proffalse

\tdtrue
%\tdfalse

\courstrue
\coursfalse


\def\classe{\textsf{PSI$\star$ -- MP}}
\def\xxnumpartie{Cycle 07}
\def\xxpartie{Modélisation des chaînes de solides dans le but de déterminer les contraintes géométriques dans les mécanismes}

\def\xxnumchapitre{Chapitre 2 \vspace{.2cm}}
\def\xxchapitre{\hspace{.12cm} Hyperstatisme}


\def\discipline{Sciences \\Industrielles de \\ l'Ingénieur}
\def\xxtete{Sciences Industrielles de l'Ingénieur}




\def\xxactivite{TD 01}
\def\xxauteur{\textsl{Xavier Pessoles}}


\def\xxtitreexo{Micromanipulateur compact pour la chirurgie endoscopique ($\text{MC}^2\text{E}$)}
\def\xxsourceexo{\hspace{.2cm} \footnotesize{Concours Commun Mines Ponts 2016}}




  
\def\xxposongletx{2}
\def\xxposonglettext{1.45}
\def\xxposonglety{20}
%\def\xxonglet{Part. 1 -- Ch. 3}
\def\xxonglet{\textsf{Cycle 07}}

\def\xxactivite{Application}
\def\xxauteur{\textsl{Pôle Chateaubriand -- Joliot-Curie}}

\def\xxcompetences{%
\vspace{-.5cm}
\footnotesize{
\textsl{%
\textbf{Savoirs et compétences :}\\
\vspace{-.2cm}
\begin{itemize}[label=\ding{112},font=\color{ocre}] 
\item \textit{Mod2.C34} : chaînes de solides;
\item \textit{Mod2.C34} : degré de mobilité du modèle;
\item \textit{Mod2.C34} : degré d’hyperstatisme du modèle;
\item \textit{Mod2.C34.SF1} : déterminer les conditions géométriques associées à l’hyperstatisme;
\item \textit{Mod2.C34} : résoudre le système associé à la fermeture cinématique et en déduire le degré de mobilité et d’hyperstatisme.
\end{itemize}}}}


\def\xxfigures{
\includegraphics[width=.55\textwidth]{images/fig_01}
}%figues de la page de garde


\def\xxpied{%
Cycle 07 -- Modélisation des chaînes de solides \\%dans le but de déterminer les contraintes géométriques dans les mécanismes\\% afin de valider leurs performances.\\
Chapitre 2 -- \xxactivite%
}

\setcounter{secnumdepth}{5}
%---------------------------------------------------------------------------


\begin{document}
%\chapterimage{png/Fond_Cin}
\input{style/new_pagegarde}
\vspace{4.5cm}
\pagestyle{fancy}
\thispagestyle{plain}


\def\columnseprulecolor{\color{ocre}}
\setlength{\columnseprule}{0.4pt} 

\ifprof
\else
\begin{multicols}{2}
\fi
\section*{Mise en situation}
\ifprof
\else
Le robot $\text{MC}^2\text{E}$ est utilisé par des chirurgiens en tant que troisième main lors de l'ablation de la vésicule biliaire. La cinématique du robot permet de garantir que le point d'insertion des outils chirurgicaux soit fixe dans le référentiel du patient. 

Le robot est constitué de 3 axes de rotations permettant de mettre en position une pince. La pince est animée d'un mouvement de translation permettant de tirer la vésicule pendant que le chirurgien la détache du foie. 


\begin{obj}
Valider par un calcul simplifié de pré-dimensionnement la motorisation de l'axe 1 du  $\text{MC}^2\text{E}$.

\end{obj}


\ifprof
\else
\begin{center}
\begin{tabular}{|p{.95\linewidth}|}
\hline
\textbf{Corrigé résumé}
\begin{enumerate}
\item $\quad$
\item $\quad$
\item $\quad$
\end{enumerate} \\
\hline
\end{tabular}
\end{center}
\fi


\ifprof
\else
\end{multicols}
\fi

\ifprof
\else

\vspace{1cm}
\begin{center}
\includegraphics[width=\linewidth]{images/fig_03}
%\textit{}
\end{center}
\fi
%\begin{center}
%\includegraphics[width=\linewidth]{images/fig_04}
%%\textit{}
%\end{center}

\end{document}

\subparagraph{}\textit{}

\begin{center}
\includegraphics[width=\linewidth]{images/img_04}
%\textit{}
\end{center}

